\chapter{The Visual Profiler Backend}
The backend is the part of the profiler resposible for collecting the data from the profiled application and for sending them to the analitic part of the Visual Profiler frontend. There are many requirement for a profiling backend to fulfill, such as speed, mutli-threadinng, correctness, low memory consumption and others. In this chapter, we will introduce the profiling modes we chose to implement and review the inner implementation of the backend and desing decision we had to make.

\section{Choice of the implentation strategy and the profiling modes}
Based on the analysis of the profiling modes on the .NET platform in the chapter \ref{chapPerProfOnDotNet}. We made a decision to use the profiling API as the base for our own profiler before the implementing everything from scratch. The profiling API offers a 	 matured API with intrinsic support of all .NET features. This helps to avoid many unnecessary implementation problems that could be encountere during an implementation of a .NET profiler entirely from beginning.

In the abstract of the thesis we have committed ourself to create a method granularity profiler supporting two distinct profiling modes for the .NET.

The choice of the sampling profiling mode was straightforward because of its stochastic nature in that it completely differs from the other two modes. 

On the other hand, the difference between the tracing and the intrumentation profiling is not that vast. They both measure exact results and put similar overhead on the profiling. However, the selected method granurality makes them both even more comparable from user's point of view. They both have their implementation pitfall and neither of them would be easier to progam that the other. In the end, we opted for the tracing profiling mode since it does not require any changes to target assemblies.

\section{Software architecture of the backend }
The backend runs in two process. The first process, written in C++ and called the Visual Profiler Backend, is the actual profiled application with the hosted profiler DLL that is loaded by the CLR and accessed by the COM interface, as described in the chapter \ref{chapPerProfOnDotNet}. The sampling and tracing profiliers are implemented as distinct COM CoClasses and only one of them can run at a time. The CoClasses share the common code base for the profiled data collection and the interprocess communication with the managed code.

The second process is programmed in C\# and called the Visual Profiler Access. It starts the profiled application, written in C++, with the desired profiling mode in a separate process, initialize the interprocess communication for transferring the profiling data and is responsible for their processing so they can be later used by an analystic and UI frontend.

The whole relation is shown on the figure \ref{fig:04softwareArchitecture}.

\begin{figure}
	\centering
		\includegraphics[scale=1]{\imagePath 04softwareArchitecture.png}
		\caption{The backend's architecture}
	\label{fig:04softwareArchitecture}
\end{figure}


\subsection{The Visual Profiler Backend }
In this part are the both chosen profiling mode implemented. It also contains bunch of supporting classes for handling metadata information for methods, classes, modules and assemblies, for caching the profiling intermediate results and for serializing the results and transmitting them over IPC to the Visual Profiler Access. The Active Template Library (ATL) framework from Microsoft is used to simplify programing of the Component Object Model (COM) in C++

Let us now introduce you the inner structure and rough outline.

\subsubsection{Implentations of the \textit{ICorProfilerCallback3 } interface}
The \textit{ICorProfilerCallback3} interface is the center point of the profiling API. With its 80 methods it is a little bit unpractical to implement it, mainly because only a handful of the method is usualy made use of. For that reason we created a default class \textit{CorProfilerCallbackBase} as depicted on the figure \ref{fig:04ProfilerCallbacks}. This provides empty implementation of every methods of the interface and derived classes can only override desired methods. This idea makes it easier to have clearly arranged classes, the clean code rules.

\begin{figure}
	\centering
		\includegraphics[scale=1]{\imagePath 04ProfilerCallbacks.png}
		\caption{Implentations of the \textit{ICorProfilerCallback3} interface}
	\label{fig:04ProfilerCallbacks}
\end{figure}

$$The both classes, the \textit{TracingProfiler} and the \textit{SamplingProfiler}, override the \textit{Initialize}, \textit{ThreadCreated} and \textit{ThreadDestroyed}. This
$$
\subsubsection*{The \textit{TracingProfiler} class}
This class is home for the tracing profiler mode implementation, the figure \ref{fig:04ProfilerCallbacks}. It starts its lifetime by running the IPC with the Visual Profiler Access on a separate thread in the \textit{Constructor()} method. After that, the \textit{Initialize()} method is invoked by the CLR and passed a pointer to the \textit{ICallProfilerInfo3} interface. Here registers the profiler the function ID mapper (more detail later in this subsection), the function enter/leave, the thread created/destroyed and the exception events notifications. At this point is everything set up and the profiled application starts running.

The profiler receives a notification from the CLR whenever a new managed thread is created. The notification executes within the context of the newly created thread. The tracing profiler registers the thread id and associates with it an instance of the \textit{TracingCallTree} class. Such an association is saved into a hash table, represented by C++ \textit{std::map}, under the thread id key. The reference to the \textit{TracingCallTree} object is stored into a thread local storage in order to avoid repetitive search in the hash table by every single enter/leave notification.

The \textit{TracingCallTree} class tracks the method enter and leave notifications for every managed thread simultaneously. This is a demanding task if you take in account the every single call to and return from a profiled method is augmented by overhead of this notification. Our first approach to store the collected profiling information was to push the data into a stack. That was fairly easy to implement and did indeed worked. 
However not very well. Firstly, it used so much memory that after tens of seconds the application memory working set reached hunderts of mega bytes, secondly it slowed down the application (at that time we did not used the thread storage, so the the corresponding \textit{TracingCallTree} objected had to be found every single time in the hash table). It was clearly unacceptable. We had to come with a better solution since the profiled application was running 45 times slower (according to rough messurements of duration in the profiled application). 

We came up with a solution that is based on the fact that an union of snapshots of thread's call stacka forms a call tree as depicted on the figure \ref{fig:04PrimitiveCallTree}.

\begin{figure}
	\centering
		\includegraphics[scale=1]{\imagePath 04PrimitiveCallTree.png}
		\caption{Union of thread's snapshots forming a call tree. A very simple example: A calls  B, B calls C, C returns to B, B calls D, D returs to B, B return to A }
	\label{fig:04PrimitiveCallTree}
\end{figure}
