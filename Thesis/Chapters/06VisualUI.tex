\chapter{User Interface and Visual Studio Integration }
\label{chap:VPUIandSI}


This chapter explains the visual presentation of the profiling results in a source code editor and the integration of the profiler in Visual Studio.

\section{Mapping profiling result to source code}
The profiling results are usually presented in others profiler in great detail by tables, graphs and tree-structures. The difficulty in reading the data is coping with large volumes of data that they present. We believe developers would benefit from briefer overview mapped to context of the source code.

S. G. Eick and J. L. Steffen presented in \cite{EickSteffen92} a method for visualizing line oriented profile data. It displays files as rectangles and code as colored lines within each rectangle. The line color is determined by the profile counts. They claim that their technique allows programmers to discover usage patterns in the code that would be impossible to find using traditional methods. They implemented their ideas in a software tool Seesoft, on the figure \ref{fig:05SeeSoft}.

\begin{figure}
	\centering
		\includegraphics[scale=0.5]{\imagePath 05Seesoft.png}
		\caption{Seesoft - mapping each line of code into a thin row, colored according to a statistics of interest and showing the context in a preview window.}
	\label{fig:05SeeSoft}
\end{figure}


We followed and extended their idea and proposed and implemented a method level coloring and results overview based on profiling results from the Visual Profiler. 

\section{Source code coloring and bird-eye view}
To present the profiling results in the Visual Profiler colored rectangles are laid over methods directly in the source code editor as depicted on an example on the figure \ref{fig:05SourceCodeHighlight}. Such approach provides direct feedback to developers and revels the ''hot-spots'' in direct relation with the code in very clear way.  The color of the overlay represents the significance of the particular methods in the context of the profiling metrics.

\begin{figure}
	\centering
		\includegraphics[scale=0.7]{\imagePath 05SourceCodeHighlight.png}
		\caption{Laying a colored rectangle over the source code. The color hue of the rectangle corresponds to a measured value of the method. }
	\label{fig:05SourceCodeHighlight}
\end{figure}

The coloring of the source code provides great local awareness of the profiling result but says nothing about the overall results of a profiling session. 

We addressed this problem by introducing a ''bird-eye'' view of the source code with applied coloring. The bird-eye view consist of columns of rectangles. The columns represent individual source files and the rectangles individual methods. The height of the rectangles corresponds to the amount of the source code lines, their order in the column reflects their position in the source file and their color is set according the profiling results.
When the mouse cursor hovers over a rectangle the detail information for the corresponding method is displayed. The whole situation is illustrated on the figure \ref{fig:05BirdeyeView}.

\begin{figure}
	\centering
		\includegraphics[scale=0.7]{\imagePath 05BirdeyeView.png}
		\caption{Bird-eye view representation of source files as columns and methods as rectangles with highlighted method and detailed information. }
	\label{fig:05BirdeyeView}
\end{figure}

The direct source code overlay was missing a possibility to get more exact information about the profiling result and the rectangles from the bird-eye view were lacking a linkage to source code. So we connected them. When the mouse hovers over a source code overlay, a corresponding rectangle is highlighted and details displayed in the bird-eye and when a rectangle in bird-eye view is clicked, a corresponding source file is opened and with the right method highlighted. 

The bird-eye view also joins all methods over their source file and sort them according their measured values and presents them above the source file columns as a row of squares. The color and behaviour of the squares follows the same rules as the color of the rectangles.

Switching between the different profiling modes may be accomplished by the radio buttons in the top right corner of the bird-eye view.

The bird-eye view together with the source code overlays form a very solid and conveying way to display profiling informations. 

We also incorporated a call-graph exploration feature. It allowed to see relation among methods calls. But in the end we took it away, because the UI started to become clogged and it did not bring much value to understanding the profiling results. 

The resulting UI is result of many desing iterations and mock-ups.

\section{Implementation of the user interface}
The user interface is built on the WPF \footnote{Microsoft Windows Presentation Foundation} framework \cite{WPF4Unleashed} and follows the MVVM \footnote{Model-View-ViewModel} \cite{MVVM2011} UI-design pattern. Let us now briefly introduce WPF adn MVVM.

\subsection{WPF}
Microsoft Windows Presentation Foundation is the latest framework (as in the end of 2011) for creating user interfaces with rich user experience. It is part of .NET framework and brought a new ideology in composing visual components and thier functionality. Its main features are depicted on the figure \ref{fig:05wpfMainFeatures}.

\begin{figure}
	\centering
		\includegraphics[scale=0.7]{\imagePath 05wpfMainFeatures.png}
		\caption{Overview of the main new features of WPF \cite{WPFpage}}
	\label{fig:05wpfMainFeatures}
\end{figure}

WPF is vector base, resolution independent and hardware accelerated. It combines ordinary UI components, 2D graphics, 3D graphics and multimedia. 

Visual part and of UI components is defined in an XML based language called \linebreak XAML \footnote{eXtensible Application Markup Language} and the behaviour is implemented in a managed programming language. The code and XAML use databinding, commands and events to communicate. This architecture results in high separation of appearence and behaviour.

\subsection{Model-View-ViewModel}
This UI pattern, on the figure \ref{fig:05pattern_mvvm}, was introduced with the advent of WPF and is very often used to build low-coupled applications. It consists of a model that is not aware of the other parts, of a view that manages the user input and sends it by commands to the view-model and of view-model that implement the program logic. The view gets data from the view-model through databinding.

\begin{figure}
	\centering
		\includegraphics[scale=0.7]{\imagePath 05pattern_mvvm.png}
		\caption{Overview of the main new features of WPF \cite{WPFpage}}
	\label{fig:05pattern_mvvm}
\end{figure}

\subsection{User interface - model}
The input data for the model come from the Visual Profiler Access as call trees and method, class, module and assembly metadata. The data are converted to groups representing source files and their methods, which then bear all the profiling data. This process is depicted on the figure \ref{fig:05TransformationOfTrees}. This is a non-trivial task because the profiling data is spread across multiple call trees and their elements. The various data analysis have to be performed, for instance data aggregation over call tree elements belonging to the same method, maximum values of different metrics, redistribution of values and so on.

 \begin{figure}
	\centering
		\includegraphics[scale=1]{\imagePath 05TransformationOfTrees.png}
		\caption{Call trees and metadata are merged to form source file groups with aggregated methods.}
	\label{fig:05TransformationOfTrees}
\end{figure}

For every profiling mode (tracing or sampling) there are various criteria with diferent units to measure, which adds to the complexity (e.g. call count [hits], duration [s]...). In order to make the implementation as reusable as possible (mainly because of UI reusability and independence) we introduced abstractions of values, criteria and criteria contexts, as show on the figure \ref{fig:05UImodel}. The abstract value is capable of converting itself to a string and to a value on the scale from 0 to 1 provided the maximum value. The criteria know their names and units. The criteria context tracks all available criteria and their maximum values. 

 \begin{figure}
	\centering
		\includegraphics[scale=1]{\imagePath 05UImodel.png}
		\caption{The class hierarchy of the model part.}
	\label{fig:05UImodel}
\end{figure}

\subsection{User interface - view-model}
The view-model is responsible for the control of the UI. It holds references to the model data and exposes them to the view. The view-model handles user input by commands that are bound to the view. It controls the view by changes of view-model properties that are databound to dependency properties \footnote{Dependency property - \href{http://msdn.microsoft.com/en-us/library/ms752914.aspx}{http://msdn.microsoft.com/en-us/library/ms752914.aspx}} of the view.

Every view-model inherits from the \textit{ViewModelBase} class, the figure \ref{fig:05ViewModelBase}. This class implements the \textit{INotifyPropertyChanged} interface that allows the WPF databinding to re-databind whenever a view-model property changes. When a view-model property changes, it calls in its setter the \textit{ViewModelBase}'s \textit{OnPropertyChanged()} method and passes its name as a parameter and the \textit{ViewModelBase}'s \textit{OnPropertyChanged()} method in turn invokes the \textit{ViewModelBase}'s \textit{PropertyChanged} event, which notified the registered handlers (presumably some WPF delegates) about the change to the property. This mechanism is an elegant way to achieve loose coupeling between view and view-model.
 
 \begin{figure}
	\centering
		\includegraphics[scale=1]{\imagePath 05ViewModelBase.png}
		\caption{The \textit{ViewModelBase} class implements \textit{INotifyPropertyChanged} interface.}
	\label{fig:05ViewModelBase}
\end{figure}

The middle point of the UI logic is the class \textit{UILogic}. This class holds references to all view-models, handles actions from commands and updates the views-models, which consequently update the views.

The other parts of the view-model are:
\begin{itemize}	
\item  \textit{ContainingUnitViewModel} - represents a source file and holds \textit{MethodViewModel}s, this view-model is shared by three views	

\item \textit{CriterionSwitchViewModel} - provides data and commands for switching profiling criteria

\item \textit{DetailViewModel} - holds data for a method in focus

\item \textit{MethodViewModel} - exposes data and commands for highlighting and changing information in the UI, this view-model is shared by three views	
\end{itemize}


\subsection{User interface - view}
The view is based on a compostion of user controls. Dynamic display is accomplished by WPF panels. The placement of view in the UI isillustrated on the figure \ref{fig:05UIViewsArrows}. 

The views are:
\begin{itemize}	
\item  \textit{ContainingUnitView} - contains methods, this view has no visual parts, it provides only layout 

\item \textit{CriteriaSwitchView} - holds the radio buttons for switching the profiling criteria

\item \textit{DetailView} - displays data for a method in focus

\item \textit{MethodView} - represents a method in source file, invokes MouseUp, MouseEnter, MouseLeave events

\item \textit{SortedMethodsView} - is a container for all methods sorted by an actual criterion

\item \textit{SortedMethodView} - same functionality as \textit{MethodView}

\item \textit{VisualProfilerUIView} - the top level container, encompasses all other controls

\item \textit{VSPackage.ContainingUnitView} - identical functionality to \textit{ContainingUnitView}, placed in the source editor 

\item \textit{VSPackage.MethodView} - like \textit{MethodView}, resides in the source editor

\end{itemize}

Every view corresponds to one view-model. However, a view-model might be shared by multiple views. This can save a lot of development effort and coding. The table \ref{tab:05viewviewmodels} expresses the relation among views and view-models in the UI.

\begin{figure}
	\centering
		\includegraphics[scale=0.5]{\imagePath 05UIViewsArrows.png}
		\caption{Placement of the views in the final UI look.}
	\label{fig:05UIViewsArrows}
\end{figure}

\begin{table}
\centering
    \begin{tabular}{|l|l|}
        \hline
        \textbf{Views}                                                      & \textbf{View-models }    \\ \hline
        ContainingUnitView 													&       				   \\ 
        SortedMethodsView 													& ContainingUnitViewModel  \\ 
        VSPackage.ContainingUnitView 										&   					   \\ \hline
        CriteriaSwitchView                                                  & CriterionSwitchViewModel \\ \hline
        DetailView                                                          & DetailViewModel          \\ \hline
        MethodView										                    &                          \\
        SortedMethodView								                    & MethodViewModel          \\
        VSPackage.MethodView                                                &                          \\ \hline
    \end{tabular}
    \caption{The relation among views and view-models in the UI.}
   	\label{tab:05viewviewmodels}
    	
\end{table}


\section{Visual Studio 2010 IDE integration}
The Visual Studio 2010 IDE is a composition of a shell and packages - functional units. The shell provides graphical user interface and services for the packages. The core functionality, project types, testing, compilation services and many more, of the IDE is implemented in the packages. Many third-party extensions of the Visual Studio are also packages.

Visual Studio offers numerous ways to extend it. The philosophy behind the extensibility approaches is fragmented, mainly due to historical design decisions, different technologies used and backwards compatibility. The complexity of it is briefly depicted on the figure \ref{fig:05VSExtensibilityGraph}. Finding the right interface to use is sometimes real challange. There is not services discoverability and everything has to be search in the scattered documentation and samples.


 \begin{figure}
	\centering
		\includegraphics[scale=0.5,angle=90]{\imagePath 05VSExtensibilityGraph.png}
		\caption{Main extension points of Visual Studio IDE. }
	\label{fig:05VSExtensibilityGraph}
\end{figure}

There are many ways how to add new functionality to the IDE. We will briefly summarise them and then delve into details of the add-ins and MEF components. The following approaches may be freely combined.

\subsection{Macros}
Macros are ability to add function by composing existing functionaly. The functionality in Visual Studio is exposed by commands and  there is a macro language similar to Visual Basic that allows to executing the commands and register new ones. Macros save repetition - replacing click and dialogs with a single key stroke can save a lot of time.

\subsection{Snippets}
Snippets do not add new capabilities to Visual Studio, they just save typing and thus error rate. For instance the snippet of a dependency property is invaluable.

\subsection{Templates}
Templates organize and encompass project definitions. When you bring up a new project dialog in Visual Studion. What it is showing you is all the templates it know about. When you make a choice, the template gets used and all the information like what files to start with, what refences will be needed in your project, setting of builds - all that is in template. Templates are Zip files

\subsection{Editor Extensions - MEF}
We have to cite the MEF's website because it describes MEF in very elaborate way:

''The Managed Extensibility Framework (MEF) is a composition layer for .NET that improves the flexibility, maintainability and testability of large applications. MEF can be used for third-party plugin extensibility, or it can bring the benefits of a loosely-coupled plugin-like architecture to regular applications.''\cite{MEFpage}

Actually MEF 
\footnote{Managed Extensibility Framework - \href{http://mef.codeplex.com/}{http://mef.codeplex.com/} } 
is basicly an enhanced dependency injection system. 

The new Visual Studio editor is written in WPF and chose the MEF as an extensibility platform. MEF is used exclusively for extending only the editor and modeling and diagraming tools (Ultimate edition)!

The editor offer many extension points as e.g. content types (code completion, syntax coloring), classification types and formats (key words, syntax errors, comments and so on), tags, margins and scrollbars, \textbf{adornments} (any visual effects in the editor) - used in the UI to color methods, mouse processors and drop handlers and IntelliSence. The samples provided by Visual Studio SDK are very good starting point for experimenting with the extensibility
\footnote{Visual Studio Software Development Kit - \href{http://msdn.microsoft.com/en-us/library/bb166441(v=VS.100).aspx}{MSDN website}}.

Editor extension are deployed as Visual Studio packages.

\subsection{Add-In}
An add-in is a compiled DLL that runs in the Visual Studio and contains a class that implements \textit{IDTExtensibility2} - COM interface. It is registered with Visual Studio by an XML file. At load time the add-in receives the Application object, \textit{DTE2} object, that represents Visual Studio and it can that call method on that object to interact with the IDE. An add-in has very broad scope than e.g. an editor extension. You can interact with almost everything in Visual Studio.

\subsection{Package}
A package is a compiled DLL (native or managed) and has to implement with the \textit{IVsPackage} interface which provides it access to Visual Studio API. Managed Package Framework simplifies package development by providing standard implementations of functions of the \textit{IVsPackage} interface. It uses attributes to control registration.

The main differences between add-ins and packages are:
\begin{itemize}	
\item  add-ins were available in the Visual Studio from the very beginning as an extension mechanism and are built on COM components. Packages should overcome some limitation of add-ins and are a newer technology.

\item packages are completely integrate in the IDE whereas add-ins are external item for the IDE  .

\item packages have lower API access oppose to add-ins.

\item the package development kit supports less languages than the one of add-ins. Add-ins may be developed in any COM enabled environment.

\item VSIX files (zip files) are used to deploy packages. They are self-containing and well integrated with the Visual Studion extension manager. VSIX files are shared on the Visual Studion Gallery web site. \footnote{\href{http://visualstudiogallery.msdn.microsoft.com/}{http://visualstudiogallery.msdn.microsoft.com/}}
. Add-ins do not have a standatized deployment model and a development team has to take care for installing and registering an add-in.
\end{itemize}


\section{Visual Profiler itegration in Visual Studio}
The integration of the Visual Profiler in Visual Studio has tree parts, \textit{Tools} menu commands for interacting with the profiler, a tool window for hosting the UI and an editor adornment extension for the code coloring. The parts are packed in a single VSIX package.

\subsection{Commands}
The Visual Profiler adds three command to the \textit{Tools} menu as shown the figure \ref{fig:05ToolMenu}. The commands are


\begin{itemize}
\item Start Visual Profiler in Tracing Mode - builds the current solution, runs the start-up project with the attached profiler in the tracing mode and opens the tool windows with the profiler UI.

\item Start Visual Profiler in Sampling Mode - does the same as the previous point, only  starts the profiler in the sampling mode.

\item Stop Visual Profiler - closes the profiled process and removes the code coloring.
\end{itemize}

 \begin{figure}
	\centering
		\includegraphics[scale=0.7]{\imagePath 05ToolMenu.png}
		\caption{Commands in the \textit{Tools} menu in Visual Studio 2010. }
	\label{fig:05ToolMenu}
\end{figure}

\subsection{Tool windows}
The tool window is shown right after the profiler starts. The result are updated already during profiling in a second long intervals. The tool window and its functionality was already described earlier in the chapter and depicted on the figures \ref{fig:05BirdeyeView} and \ref{fig:05UIViewsArrows}.

\subsection{Editor adornment extension}
The editor extension is using MEF to add a canvas on an editor's adornment layer for every file encountered during the profiling. The rectangles representing methods and their profiling results are drawn on the canvas. The data is live updated in sync with the main UI. The visual elements are shown on the figures \ref{fig:05SourceCodeHighlight} and \ref{fig:05UIViewsArrows}.












