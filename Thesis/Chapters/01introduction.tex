\chapter{Introduction}

According the Pareto's law (also known as 80/20 rules) 80 percent of the results comes from 20 percent of the effort \cite{RicKoch1999}. This law is applicalbe to software development. 80 percent of all end users generally use only 20 percent of a software application's features. Microsoft reported that 80 percent of errors and crashes in Windows and Office are debited to 20 percent of bugs \cite{PauRoon2002}.

The same principle applies to software performace. 80 percent of time is spent in 20 percent of code. Some argue that it is even more, 90/10. So, investing programmers' effor to the 20 percent of code may have great effect on the overall speed - if that 20 percent of code be discovered.

Programmers are not particularly successful at guessing which part of the code is crutial for the perfomance \cite{SteMcCo2004}. Therefore profilers help to automate the search of bottleneck and hotspots in applications and their source code and provide valuable metrics, such as execution times of specific part of code or memory usage of a given object. .NET programs' performance profiling ist the main area of this thesis.

The performace profiling dynamically analysises execution behaviour of a program. It tracks various runtime related data as frequency and duration of function calls. Analysis and visualition of the gathered data provides useful hint on the program's code runtime characterics and helps during optimalization and exploration of the program.
	
Profiling can be achived by various means as e.g. by inserting tracing code into either the source code or the binary executalbe of the program or by runtime sampling of thread call stacks of the program or by listening to events invoked by a program's runtime engine.

Each of aforementioned approaches differs in overhead imposed to the program and in kind, precision and granuality of the gathered data.

In the world of .NET performace profiling exist already few full-fledged solutions targeting almost all .NET platforms from desktop to Windows Phone applications. However, they are mostly commercial and do not provide deep integration into development tools.  

In this thesis, we will to introduce a development-time .NET profiler offering various profiling methods and allowing direct interaction directly from the Microsoft Visual Studio 2010.