\chapter{Introduction}

According the Pareto's law (also known as 80/20 rules) 80\% of the effects comes from 20\% of the cause \cite{RicKoch1999}.

Performace profiling dynamically analysises execution behaviour of a program. It tracks various runtime related data as frequency and duration of function calls. Analysis and visualition of the gathered data provides useful hint on the program's code runtime characterics and helps during optimalization and exploration of the program.
	
Profiling can be achived by various means as e.g. by inserting tracing code into either the source code or the binary executalbe of the program or by runtime sampling of thread call stacks of the program or by listening to events invoked by a program's runtime engine.

Each of aforementioned approaches differs in overhead imposed to the program and in kind, precision and granuality of the gathered data.

In the world of .NET performace profiling exist already few full-fledged solutions targeting almost all .NET platforms from desktop to Windows Phone applications. However, they are mostly commercial and do not provide deep integration into development tools.  

In the 